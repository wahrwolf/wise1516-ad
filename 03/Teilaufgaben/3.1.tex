\paragraph{ALGO1}
Für den ersten Algorithmus gilt:

Die äußere Schleife wird n mal betreten.

Die inneren beiden Schleifen werden jeweils n mal betreten.

Daraus folgt eine gesamte Laufzeit von $(n*(n+n))$ was in $O(n^2)$ liegt.


\paragraph{ALGO2}
Für den zweiten Algorithmus gilt:

Die äußere Schleife wird n mal betreten (Grenze bei 2n, aber Schritt bei n+=2).

Die innere Schleife wird n mal ausgeführt (Schleife wird $2 \cdot n$ mal betreten, geht aber nur bis $ \frac{n}{2} $)

Daraus folgt eine gesamte Laufzeit von $(n \cdot (2 \cdot \frac{n}{2}) ) \in O(n^2)$.


\paragraph{ALGO3}
Für den dritten Algorithmus gilt:

Die äußere Schleife wird $ \sqrt{n} $ mal betreten, was aus der Äquivalenzumformung von $i \cdot i < n $ folgt.

Die innere Schleife wird $ \log_{2}{n} $ mal betreten, was aus der Halbierung von j, das mit n initialisiert wird, bei jeder Iteration folgt.

Daraus folgt eine gesamte Laufzeit von $ \sqrt{n} \cdot \log_{2}{n} $.
