$T_1(n)\ :=\ \bigg\{$
\begin{tabular}{ll}
$c_1$,&für $n = 1$\\
$8 \cdot T_1(\frac{n}{2}) + d_1 \cdot n^3,$ &sonst
\end{tabular}\\
mit $a = 8$ und $b = 2$ ist $\log_b(a) = 3$ und $f(n) = d_1 \cdot n^3$\\
Unter Verwendung des 1. Falles des Mastertheorems gilt:\\
$T(n) \in  \Theta(n^{log_b(a)}),$ falls $f(n) \in O(n^{log_b(a) - \epsilon})$ für ein $\epsilon > 0$.\\
Dies gilt hier leider nicht, da für kein $\epsilon > 0$ die Ungleichung $d_1 \cdot n^3 \leq n^{3 - \epsilon}$ erfüllt ist, beziehungsweise nur für bestimmte $d_1$.\\
Betrachtung des 2. Falles des Mastertheorems:\\
Es gilt: $T(n) \in  \Theta(n^{log_b(a)}\cdot log_2(n)),$ falls $f(n) \in \Theta(n^{log_b(a)})$\\
Da für $\lim_{n\to\infty}\frac{f(n)}{n^{log_b(a)}} = \lim_{n\to\infty}\frac{d_1\cdot n^3}{n^{3}} = \lim_{n\to\infty}\frac{d_1\cdot 1}{1} = d_1$ die Beziehung $f(n) \in \Theta(n^{log_b(a)})$ erfüllt ist. Damit gilt $T_1(n) \in \Theta(n^{log_b(a)}\cdot log_2(n))$.

$T_2(n)\ :=\ \bigg\{$
\begin{tabular}{ll}
$c_2$,&für $n = 1$\\
$5 \cdot T_2(\frac{n}{4}) + d_2 \cdot n^2$,&sonst
\end{tabular}\\
mit $a=5$ und $b=4$ ist $log_b(a) \approx 1,16$ und $f(n) = d_2 \cdot n^2$\\
Unter Betrachtung des 3.Falles des Mastertheorems gilt:\\
$T(n) \in \Theta(f(n))$, falls $f(n)\in \Omega(n^{log_b(a)+\epsilon})$ für ein $\epsilon > 0$ und $a \cdot f(\frac{n}{b})\leq \delta \cdot f(n)$ für ein $\delta < 1$ und große $n$.\\
Mit $\epsilon = 0,5$:\\
$\lim_{n\to\infty}\frac{f(n)}{n^{log_b(a)+\epsilon}} = \lim_{n\to\infty}\frac{d_2\cdot n^2}{n^{1,16 + 0,5}} = \lim_{n\to\infty}\frac{d_2\cdot n^2}{n^{1,66}} = \lim_{n\to\infty}\frac{d_2\cdot n^{0,34}}{1} = \infty$\\
Hiermit ist $f(n) \in \Omega(n^{log_b(a) + \epsilon})$.\\\\
$T_3(n)\ :=\ \bigg\{$
\begin{tabular}{ll}
$c_3$,&für $n = 1$\\
$6 \cdot T_3(\frac{n}{3}) + d_3 \cdot n \cdot \log n$,&sonst
\end{tabular}\\
mit $a = 6$ und $b = 3$ ist $log_b(a) \approx 1,63$ und $f(n) = d_3 \cdot n \cdot \log n$\\
Unter Betrachtung des 1. Falles des Mastertheorems gilt:\\
$T(n) \in  \Theta(n^{log_b(a)}),$ falls $f(n) \in O(n^{log_b(a) - \epsilon})$ für ein $\epsilon > 0$.\\
Wir wählen $\epsilon = 0,13$ und wollen damit obiges zeigen.\\
$\lim_{n \to \infty} \frac{n \cdot log_2(n)}{n^{1.63 - 0.13}} = \lim_{n \to \infty} \frac{n \cdot log_2(n)}{n^{\frac{3}{2}}} = \lim_{n \to \infty} \frac{log_2(n)}{n^{\frac{1}{2}}}$\\
Unter Anwendung der Regel von L'Hospital erhalten wir folgendes:\\
$\lim_{n \to \infty} \frac{\frac{1}{n \cdot ln(2)}}{\frac{1}{n^{\frac{1}{2}}}} = \lim_{n \to \infty} \frac{1}{n^{\frac{1}{2}}}\cdot ln(2) = 0$\\
Damit ist $f(n) \in O(n^{log_b(a) - \epsilon})$, wodurch $T_3(n) \in \Theta(n^{log_b(a)})$ gilt.

