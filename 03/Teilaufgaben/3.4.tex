\paragraph{Implementation der merge-Funktion}
\begin{verbatim}
// Der Arrayindex beginnt bei 1.
merge(A,l,q,r)
    int curA = l;
    int curB = q+1;
    Array C = new Array;
    int curC = 1;

    while ((q - curA) > -1 && (r - curB) > -1)
        if (A[curA] <= A[curB])
            C[curC] = A[curA]
            ++curA
        else
            C[curC] = A[curB]
            ++curB
        ++curC

    if ((q - curA) == -1)
        while((r - curB) > -1)
            C[curC] = A[curB]
            ++curB
            ++curC
    else
        while((q - curA) > -1)
            C[curC] = A[curA]
            ++curA
            ++curC

    return C;
\end{verbatim}

\paragraph{Schleifeninvariante}


