Beweis:
Um zu zeigen dass es stets mindestens einen transitiven Schlüsselpunkt (also einen Knoten über den t erreichbar ist) gibt, führen wir einen  Gegenbeweis.

Angenommen es existiert kein Knoten der zwingend in allen Verbindungen von s nach t vorhanden ist.

Dann muss es also mindestens 2 disjunkte Pfade von s nach t geben.

Jeder dieser Pfade benötigt mindestens $ \frac{n}{2} +1 -1 = \frac{n}{2} $ zusätzliche Knoten (neben dem Start und dem Ziel Knoten).

Für 2 Pfade brauchen wir also insgesamt $ 2 \cdot (\frac{n}{2}) + 2 = n +2$ Knoten.

Dies ist ein Widerspruch, da der Graph lediglich n Knoten hat.

Also muss es immer einen Knoten v geben der in allen Verbindungen von s nach t vorhanden ist.



Ein Algorithmus, der v findet, benötigt den Graphen $G=(V, E)$ und alle Pfade zwischen s und t aus V, für die die Bedingung $d(s,t) > n/2$ gilt.

Dann kann der Algorithmus jeden der Knoten von den Pfaden einzeln testweise löschen. Falls er anschließend keinen Pfad mehr zwischen s und t findet, hat er einen entsprechenden Knoten v gefunden. Sonst muss er den gelöschten Knoten wieder hinzufügen und den nächsten ausprobieren.

Die Korrektheit eines solchen Algorithmus ergibt sich daraus, dass jeder Knoten, der potenziell die beschriebenen Eigenschaften hat, überprüft wird. Außerdem existiert immer mindestens ein Knoten mit diesen Eigenschaften, wie wir oben gezeigt haben.

Da der zu untersuchende Graph zwischen s und t maximal n-2 Knoten hat, muss der Algorithmus maximal n-2 Knoten prüfen, was zu einer Laufzeit von $O(n)$ führt. Einen linearen Zusatzaufwand verursacht das Löschen und Wiederherstellen (auch hier werden maximal $(n-2)+(n-3)$ Operationen ausgeführt), wodurch sich insgesamt eine Laufzeit in $O(n+m)$ ergibt.

