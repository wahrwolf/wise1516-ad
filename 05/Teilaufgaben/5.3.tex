Beweis:
Um zu zeigen dass es stets mindestens einen transitiven Schlüsselpunkt (also einen Knoten über den t erreichbar ist) gibt, führen wir einen  Gegenbeweis.

Angenommen es existiert kein Knoten der zwingend in allen Verbindungen von s nach t vorhanden ist.

Dann muss es also mindestens 2 disjunkte Pfade von s nach t geben.

Jeder dieser Pfade benötigt mindestens $ \frac{n}{2} +1 -1 = \frac{n}{2} $ zusätzliche Knoten (neben dem Start und dem Ziel Knoten).

Für 2 Pfade brauchen wir also insgesamt $ 2 \cdot (\frac{n}{2}) + 2 = n +2$ Knoten.

Dies ist ein Widerspruch, da der Graph lediglich n Knoten hat.

Also muss es immer einen Knoten v geben der in allen Verbindungen von s nach t vorhanden ist.

---

Algorithmus:
Führe eine Tiefensuche von s nach t durch.

	Beachte folgende Sonderregeln:

	Bei jeder Abzweigung erhöhe einen Zähler
	Markiere den Elternknoten der Abzweigung mit aktuellem Zählerstand


Gehe den Pfad von s nach t durch:
	Ist der Knoten mit einem Zähler beschriftet?
		Reduziere Zähler um 1
	
		Ist eines der End-Elemente Teil des Erfolgspfads?
			Markiere Knoten im Erfolgspfad mit akutellem Zähler
		Sonst:
			Reduziere alle Markierungen nach dem Einstiegsknoten um 1

Hat t einen einzigen Nachbarn?
	der Nacbhar ist v
Sonst:
	der erste Knoten auf dem Pfad t -> s der Sprungmarkierung von 1 hat ist v

Begründung:

Laufzeit:

Der Algorithmus besteht aus 3 Teilen:
	Die Tiefensuche mit einer Laufzeit von $ n + |V|= $ 
		- Die Tiefesuche liegt in $ O(n^{2})$ 
		- Für die Anzahl der Kanten siehe unten
	
	Die Markierung hat eine Laufzeit von n
		- Jeder Knoten wird maximal 1x betrachtet

	Die Prüfung hat eine Laufzeit von max $\frac{n}{2} - 1 $
		- Jeder Knoten im kürzten Pfad wird maximal 1x betrachtet

Also ergibt sich insgesammt eine Laufzeit die in $O(n^{2})$ liegt.

