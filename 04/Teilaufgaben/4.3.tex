\subsection*{1. 2-SAT}

2-SAT: Menge aller aussagenlogischen Formeln, die mindestens zwei erfüllende Belegungen haben \\

Zu zeigen: 2-SAT $\in$ NPC \\

Vorgehen: Abbilden auf bereits vorhandenes NPC-Problem SAT \\

SAT ist die Menge aller aussagenlogischen Formeln die erfüllbar sind, die also mindestens eine erfüllbare Belegung besitzen. Von SAT wissen wir, dass es ein NP-vollständiges Problem ist.\\

Jetzt ist 2-SAT lediglich eine Repräsentation dieses Problems mit einem zusätzlichen Zähler, um festzustellen, ob eine Formel mindestens 2 erfüllbare Belegungen besitzt. Dadurch entscheiden sich die Probleme nur gering voneinander und wir stellen fest, dass 2-SAT, ebenso wie SAT ein NP-vollständiges Problem ist.

\subsection*{2. MINES}
$MINES$ = \{ $\langle$ $G$, $C$, $K$, $M$ $\rangle \mid$
\begin{tabular}{l}
$G = (V,E)$ ist ein ungerichteter Graph,\\
$C \subseteq V$ ist Menge der beschrifteten Knoten,\\
$K \subseteq \mathbb{N}$ ist Menge der Beschriftungen der Knoten\\
von $C$ mit $k_i \in K$ ist Beschriftung für $v_i \in C$,\\
$M \subset V$ ist Menge der Knoten, die eine Mine\\
enthalten, wobei jeder Knoten $v_i \in C$\\
$k_i$ benachbarte Knoten aus $M$ hat
\end{tabular}\}\\\\
\textbf{Beweis für $MINES \in NPC$:}\\
Wir raten, für einen in $C$ befindlichen Knoten, in welchen benachbarten Knoten, Minen seien müssen, damit die Beschriftungen der nachfolgenden Knoten aus $C$ nicht verletzt werden. Dies setzen wir für alle Knoten in $C$ fort. Hiermit gilt $C \in NP$.\\\\
Wir wollen nun SAT auf unser Problem reduzieren.\\
