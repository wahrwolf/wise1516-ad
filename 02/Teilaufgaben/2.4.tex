\subsubsection*{2.4.1}

\subsubsection*{Substituionsmethode}

\begin{tabular}{lcl}
$T(n) $	&=&$ 3 ( T(n-1) ) + 2$ \\
	&=&$ 3 ( 3(T(n-1-1) +2 ) +2 = 9 T(n-2) + 3 \cdot 2 + 2$ \\
	&=&$ 3 ( 9T(n-2-1) + 3 \cdot 2 + 2) + 2 = 21 \cdot T(n-3) 3^2 \cdot 2 + 3 \cdot 2 + 2$ \\
	&=& ... \\
(vermutlich) &=&$ 3^k \cdot T(n-k) + \Sigma_{i=0}^{k-1} $\\

\end{tabular}


Um die Vermutung zu bewweisen führen  wir eine vollständige Induktion durch.

Dabei nehmen wir an dass: $3^k \cdot T(n-k) + \Sigma_{i=0}^{k-1}$ gilt.

Die Funktion T(n) schränkt den Gültigkeitsbereich nicht ausreichend ein, da für negative Werte für n die Rekursion nie abbricht.

Für die Induktion nehmen wir an, dass $n \in \mathbb{N}$ liegt.

Daher ist unser Induktionsanfang auch bei $n=1$.

Dann gilt:

\begin{equation*}
T(1)=3 \cdot T(1-1) + 2 = 3 \cdot 0 + 2 = 2
\end{equation*}

bzw mit unserer Annahme:

\begin{equation*}
T(1)=3^1 \cdot T(1-1) + \Sigma_{i=0}^{k-1} = 3 \cdot 0 + \Sigma_{i=0}^{1-1}= 0 + 2
\end{equation*}

Damit haben wir den  Induktionsanfang gezeigt.

\begin{tabular}{lcl}
$T(n+1)								$ &=&$ T(n+1)$ \\
$3(T(n+1-1))+2							$ &=&$ 3^{k+1} \cdot T(n-(k+1) +  \Sigma_{i=0}^{k-1+1} (3^i \cdot 2)$ \\
$3(3^{k} \cdot T(n-k) +  \Sigma_{i=0}^{k-1} (3^i \cdot 2)	$ &=&$ 3^{k+1} \cdot T(n-(k+1) +  \Sigma_{i=0}^{k-1+1} (3^i \cdot 2)$ \\
\end{tabular}


\subsubsection*{2.4.2} 
Ermitteln der Groessenordnung mittels des Mastertheorems:
 
$ S(n) :=  \{ $
\begin{tabular}{lr}
$c,$&$ fuer\ n = 1;$\\
$16 \cdot S(\frac{n}{4}) + n^2,$&$ sonst; $ \\
\end{tabular}\\\\
Fall 1: $S(n) \in \Theta(n^{\log_b(a)}),\ falls\ f(n) \in O(n^{\log_b(a)-\epsilon})\ fuer\ ein\ \epsilon>0$\\
$Wegen\ \log_b(a)\ mit\ b=4\ und\ a=16 \to \log_4(16)=2\ und\\ f(n)=n^2\ ist\ f(n) \notin O(n^{2-\epsilon}),\ da\ mit\ \epsilon > 0\ O(f(n)) > O(n^{2- \epsilon})$\\\\
Fall 2: $S(n) \in \Theta(n^{\log_b(a)}\log_2(n)),\ falls f(n) \in \Theta(n^{log_b(a)})$\\
$mit f(n) und \log_b(a) wie oben ist f(n) \in \Theta(n^2)$\\
$\lim_{n->\infty} \frac{n^2}{n^2} = \lim_{n->\infty} 1 = 1,\ da\ 0 < 1 < \infty \to f(n) \in \Theta$\\\\
Hiermit gilt $T(n) \in \Theta(n^{\log_b(a)}\log_2(n))$

