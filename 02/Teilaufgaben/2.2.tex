\begin{tabular}{lcl}
Funktion				& $\in $ 		& Äquivalenzklasse 		\\
\{4, 1000\}				& $\subset$ 	& O(1) 					\\
\{$ln(n), log(n)$\} 	& $\subset$		& O(log(n)				\\
$n^{0.5}$				& $\in$			& O($\sqrt{n}$)			\\
$\sqrt{n}^{3}$			& $\in$			& O($n^{\frac{3}{2}}$)	\\
$n^{2}$					& $\in$			& $O(n^2$				\\ 
$2^{n}$					& $\in$			& $O(2^n$				\\ 
\end{tabular}

Die obige Tabelle zeigt das Wachstumsverhalten in aufasteigender Reihenfolge.
Funktionen die in der gleichen Äuqivalenzklasse liegen sind entsprechend geklammert.
Es folgt eine Begründung der Zusammenfassungen.

\begin{tabular}{l|r}
Funktion				& Begründung \\
\{4, 1000\}				& Beide überschreiten nie einen konstanten Wert \\
\{$ln(n), log(n)$\} 	& Das Wachstum ist bis auf einen konstanten Faktor gleich \\
$n^{0.5}$				&	\\
$\sqrt{n}^{3}$			& Nach den Potenzgesetzen gilt $\sqrt{n}^a = n ^{\frac{a}{2}} $\\
$n^{2}$					& \\ 
$2^{n}$					& \\ 
\end{tabular}

