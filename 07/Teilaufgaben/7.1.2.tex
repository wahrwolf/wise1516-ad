\subsection*{7.1.2: Beweis, dass $f(u,v)=f(v,u)=0$ gilt}

Die Funktion $f/2$ gibt an wie gro� die maximal zu �bermittelnde Kapazit�t zwischen zwei Knoten sein kann.

Die Aussage "es gibt keine Kante zwischen $u,V$ ist gleichbedeutend mit der Aussage "alle Kanten zwischen $u,v$ haben die Kapazit�t 0" bei Betrachtung eines Graphen als Flusses.

Wenn es zwischen $u,v$ keinen Austausch gibt bedeutet dass aber auch dass sich an dem gesamt Fluss nichts �ndern kann wenn man die Kapazit�t einer Verbindung verdoppelt.

Also muss $|f| + f(u,v) = |f|$. Dies gilt allerdings nur wenn $f(u,v)=0$ gilt.

Per Definition gilt au�erdem $f(u,v)=-f(v,u) = 0$.
